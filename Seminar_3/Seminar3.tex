\documentclass{article}[12pt]
\usepackage{amsmath}
\usepackage{amsfonts}
\usepackage{graphicx}
\usepackage{enumerate}
\usepackage{dtklogos}
\usepackage{verbatim}
\usepackage{url}
\usepackage{natbib}
\usepackage{calrsfs}
\usepackage{collectbox}
\usepackage{blindtext}
\newcommand{\R}{{\mathbb R}}
\renewcommand{\vec}[1]{{\mathbf #1}}
\newcommand{\points}[1]{\phantom{.}\hfill \textbf{(#1 points)}}
\newcommand{\matlab}{{\textsc{Matlab}} }


\begin{document}
\begin{center}


\title{Resampling Method - Seminar 3}
\hfill Iliass Tiendrebeogo\\
\hfill ID: 30742742

\end{center}
\bigskip

\begin{center}
  \begin{Large}
      
    Resampling Method - Seminar 3 \\
    Math 567: Winter 2016 \\
 
   \small \today\\
    
  \end{Large}
\end{center}

\bigskip

{\bf \large The \verb|jackknife| Resampling Method}

\bigskip 
The {\bf jackknife} was a forerunner of the bootstrap, developed by
Maurice Quenouille in 1949 to estimate the {\bf bias} of an estimator
$\hat \theta$ of a population parameter $\theta$, and extended (and
given its name) by John Tukey in 1958 to estimate the {\bf standard
error} of $\hat \theta$. \cite{David}
\smallskip
\smallskip 

\begin{enumerate}
\item % Mathematic
{\bf Estimation}

To find the {\bf \verb|jackknife|} estimate of a parameter, we estimate the parameter for each subsample omitting the $ith$ observation to estimate the previously unknown value of parameter $\bar{x_i}$
$$ \bar{x}_i = \frac{1}{n-1}\sum_{j\neq i}^{n} x_j $$
\bigskip
{\bf Variance Estimation}

An estimate of the variance of an estimator can be calculated using the jackknife technique.
$$ \mathrm{Var_{(jack)}} = \frac{n-1}{n} \sum_{i=1}^{n}(\bar{x}_{i} - \bar{x}_\mathrm{(.)})^{2}   $$

 $\bar{x}_i$ is the parameter estimate based on leaving out the ith observation, and $\bar{x}_\mathrm{(.)}$ is the estimator based on all of the subsamples.

\bigskip
{\bf Bais Estimation}


$$\widehat{Bais}_{jack} =(n-1)(\hat{\theta}_{(.)} - \hat{\theta})  $$


{\bf first order estimate of } $\theta$ is 
$$   \widehat{\text{Bias}}_\mathrm{(\theta)}=n\hat{\theta} - (n-1)\hat{\theta}_\mathrm{(.)} $$

The  multiplative factor $(n-1)$ is chosen to make this work out exactly right for the case of estimating the variance. \cite{Kate}

\bigskip 
{\bf \large The Relationship Between the Jackknife and the Boot-
strap}

The jackknife is like a bootstrap in which sampling is done without replacement
instead of with, and the samples are of size $(n − 1)$ instead of $n$.

\bigskip

unless $n$ is large, jackknife is less computationally intensive. But if jackknife jackknife uses fewer samples than bootstrap, then jackknife is using less information. 

\item % Application
{\bf \large  The jackknife estimate of bias of our dataset using R language }


{\bf We are computing the ackknife estimate of bias of the median }

\bigskip
First we install and load the \verb|"bootrap"| package \\

\verb|install.packages("bootstrap")|\\
\verb|library(bootstrap)| \\

load the data\\

\verb|data <- read.csv('Seminar_2.csv', header = TRUE, sep = "")|\\

we extract each vector from the dataset\\

\verb|col1 <- data[[1]]|\\
\verb|col1 <- data[[2]]|\\
{\bf theta function}\\
This is the theta function  which will be passed to jackknife method

\verb|theta <- function(x){ median(x) }|\\

\verb|result1 <- jackknife(col1, theta)|\\
\verb|result1$jack.bias|\\
result : \verb|>  0.1124775|\\

\verb|result2 <-jackknife(col2, theta)|\\
\verb|result2$jack.bias|\\
result : \verb|>  -0.04999|

\end{enumerate}



\bibliography{Seminar3}
\bibliographystyle{plainnat}
\end{document}