\documentclass{article}[12pt]
\usepackage{amsmath}
\usepackage{amsfonts}
\usepackage{graphicx}
\usepackage{enumerate}
\usepackage{dtklogos}
\usepackage{verbatim}
\usepackage{url}
\usepackage{natbib}
\usepackage{calrsfs}
\usepackage{collectbox}
\usepackage{blindtext}
\newcommand{\R}{{\mathbb R}}
\renewcommand{\vec}[1]{{\mathbf #1}}
\newcommand{\points}[1]{\phantom{.}\hfill \textbf{(#1 points)}}
\newcommand{\matlab}{{\textsc{Matlab}} }


\begin{document}
\begin{center}


\title{Resampling Method - Seminar 3}
\hfill Iliass Tiendrebeogo\\
\hfill ID: 30742742

\end{center}
\bigskip

\begin{center}
  \begin{Large}
      
    Resampling Method - Seminar 3 \\
    Math 567: Winter 2016 \\
 
   \small \today\\
    
  \end{Large}
\end{center}

\bigskip

{\bf \large The \verb|jackknife| Resampling Method}

\bigskip 
The {\bf jackknife} is a resampling technique developed by Maurice Quenouille (1949, 1956) and John Tukey (1958). It preceded the {\bf bootstrap } thechnique, it's most used for {\bf variance} and {\bf bias} estimation.

\begin{enumerate}
\item % Mathematic
{\bf Estimation}

To find the {\bf \verb|jackknife|} estimate of a parameter, we estimate the parameter for each subsample omitting the $ith$ observation to estimate the previously unknown value of parameter.
$$ \bar{x}_i = \frac{1}{n-1}\sum_{j}^{n} x_j $$
\bigskip
{\bf Variance Estimation}

An estimate of the variance of an estimator can be calculated using the jackknife technique.
$$ \mathrm{Var_{(jack)}} = \frac{n-1}{n} \sum_{i=1}^{n}(\bar{x}_{i} - \bar{x}_\mathrm{(.)})^{2}   $$

 $\bar{x}_i$ is the parameter estimate based on leaving out the ith observation, and $\bar{x}_\mathrm{(.)}$ is the estimator based on all of the subsamples.

\item % Application
{\bf \large The jackknife estimate of bias of our dataset using R language }

First we install and load the \verb|"bootrap"| package \\

\verb|install.packages("bootstrap")|\\
\verb|library(bootstrap)| \\

load the data\\

\verb|data <- read.csv('Seminar_2.csv', header = TRUE, sep = "")|\\

\end{enumerate}




\end{document}