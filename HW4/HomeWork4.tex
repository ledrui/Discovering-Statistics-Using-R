\documentclass{article}[12pt]
\usepackage{amsmath}
\usepackage{amsfonts}
\usepackage{graphicx}
\usepackage{enumerate}
\usepackage{dtklogos}
\usepackage{verbatim}
\usepackage{url}
\usepackage{natbib}
\usepackage{calrsfs}
\usepackage{collectbox}
\newcommand{\R}{{\mathbb R}}
\renewcommand{\vec}[1]{{\mathbf #1}}
\newcommand{\points}[1]{\phantom{.}\hfill \textbf{(#1 points)}}
\newcommand{\matlab}{{\textsc{Matlab}} }


\begin{document}

\hfill Iliass Tiendrebeogo\\

\hfill \today\\

\bigskip

\begin{center}
  \begin{Large}
    Math 567: Homework 4 \\
    
   
  \end{Large}
\end{center} 


\begin{enumerate}[1.]
\item  % 1 
{\bf The working directory }
 \begin{enumerate}[A.]
 \item 
The working directory is the default location where R will search for and save files.
\end{enumerate}
\item %2
{\bf The \texttt{subset}() function }
 \begin{enumerate}[D.]
 \item 
  Both b and c.
  \end{enumerate}

\item %3

 \begin{enumerate}[C.]
 \item 
  The \texttt{dataframe} has 2 variables.
  \end{enumerate}

\item %4
The command that would load a .CSV data file:
 \begin{enumerate}[B.]
 \item 
  \texttt{lecturerData = read.csv("Lecturer Data.csv", header = TRUE) }
  \end{enumerate}

\end{enumerate}
\end{document}