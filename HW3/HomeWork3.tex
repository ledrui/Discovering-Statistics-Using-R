\documentclass{article}[12pt]
\usepackage{amsmath}
\usepackage{amsfonts}
\usepackage{graphicx}
\usepackage{enumerate}
\usepackage{dtklogos}
\usepackage{url}
\usepackage{natbib}
\usepackage{calrsfs}
\usepackage{collectbox}
\newcommand{\R}{{\mathbb R}}
\renewcommand{\vec}[1]{{\mathbf #1}}
\newcommand{\points}[1]{\phantom{.}\hfill \textbf{(#1 points)}}
\newcommand{\matlab}{{\textsc{Matlab}} }


\begin{document}

\hfill Iliass Tiendrebeogo\\

\hfill \today\\

\bigskip

\begin{center}
  \begin{Large}
    Math 567: Homework 3 \\
    
   
  \end{Large}
\end{center} 


\begin{enumerate}[1.]
\item  % 1 
{\bf Definite and Indefinite Integrals}
$$\text{\bf Indefinite integral:} \int \! x^2 dx = \frac{1}{3}x^3 + C$$
$$\text{\bf Definite integral:} \int_0^2 \! x^2 dx = \frac{8}{3}$$

A function has many indefinite integrals which differ from each other by constants.

A definite integral of a function over a fixed interval is a number.

These two are related by the {\bf fundamental Theorem of Calculus},
$$ \int_a^b \! F'(x) dx = F(b)-F(a) \qquad(F \text{ is an antiderivative of } f)$$
$$ \int_a^x \! F'(t) dt = F(x)-F(a)$$
if this second equation is differentiate with respect to $x$, the result is (here $f=F'$)

$$ \text{Derivative of an Integral } \frac{d}{dx}\int_a^x \! f'(t) dt = f(x)$$
{\bf Anti-derivative vs Area Under Curves:}

\item 

\end{enumerate}
\end{document}