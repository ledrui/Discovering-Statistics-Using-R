\documentclass{article}[14pt]
\usepackage{amsmath}
\usepackage{amsfonts}
\usepackage{graphicx}
\usepackage{enumerate}
\usepackage{dtklogos}
\usepackage{verbatim}
\usepackage{url}
\usepackage{natbib}
\usepackage{calrsfs}
\usepackage{collectbox}
\newcommand{\R}{{\mathbb R}}
\renewcommand{\vec}[1]{{\mathbf #1}}
\newcommand{\points}[1]{\phantom{.}\hfill \textbf{(#1 points)}}
\newcommand{\matlab}{{\textsc{Matlab}} }


\begin{document}

\hfill Iliass Tiendrebeogo\\

\hfill ID: 30742742 \\

\bigskip

\begin{center}
  \begin{Large}
    Math 567: Seminar 1\\
    Jannuary 25, 2016\\
   
  \end{Large}
\end{center} 


\begin{enumerate}[1.]
\item  % 1 
{\bf \large Plot the dnorm for the range \{ -4, 4\} }


The Figure 1 shows the normal probability density function curve with the normal distribution i.e mean=0 and standard deviation=1. The probability density function of the

normal distribution is also called Gaussian or "bell curve" which is the continuous random

distribution. The probabilities of intervals of values correspond to the area under the curve. It

shows that about 95% of the distribution lies within 2 standard deviation of the mean.
\item %2
{\bf \large what is the probability if the mean is zero and normal distribution is symmetrical i.e probability(q ≤ 0) is 0.5 }

 Solution: 0.5

{\bf \large what is the probability that the value is less than 1 standard deviation below the mean?}

Solution: 0.8413447
 
\medspace

{\bf \large what is the probability that the value is within 1.96 standard deviations of the mean?}
 
 solution: 0.9500042

\item %3
 { \bf \large Plot the cumulative normal distribution for the range $\{ -4, 4\}$ }
 
 The attached Figure2 is the plot of the normal cumulative distribution function. The function evaluated at 'x' denotes the probability that a real­valued random variable X will take a value less than or equal to x, i.e $CDF(x) = Pr(X<=x)$, here Pr denotes probability.
\item %4
 {\bf \large Compute the left and right sides of a 95\% confidence interval}

Solution: 1.959964, ­1.959964

 {\bf \large Compute the median of the normal distribution}

Solution : 0

\end{enumerate}

\end{document}